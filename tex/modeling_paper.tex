\include{preface}
\title{Notes on Modeling Formaldehyde}

The \ortho \oneone and \twotwo lines have optical depths and line
brightnesses dependent on a number of different factors all to varying degrees.
In approximate order of relative importance, these are:

\begin{enumerate}
    \item Volume density ($n(\hh)$ \percc)
    \item Column Density ($N(\ortho)$ \persc) \perkms
          (degenerate with abundance $X_{\ortho}$ and velocity gradient)
    \item Ortho/Para ratio of \hh
    \item Helium abundance (similar effect as p-\hh)
    \item Temperature
\end{enumerate}


\section{Optical Depths}
The optical depths can be used to directly determine the volume and column
density of \ortho, but with major degeneracies with abundance.

\FigureThreePDF{figures/tau_vs_density_thinlimit_sigmavary_Xm10}
               {figures/tau_vs_density_thinlimit_sigmavary_Xm9}
               {figures/tau_vs_density_thinlimit_sigmavary_Xm8}
{Optical depth of \ortho \oneone and \twotwo as a function of \dens for three
different abundances.  The different $\sigma$ values represent different widths
of lognormal distributions.
(a) $X=10^{-10}$ (b) $X=10^{-9}$ (c) $X=10^{-8}$
}
{fig:tauvsdens}
.

\section{Ratios}
Optical depth ratios have much less variation with abundance and are therefore more universally
robust, but there is only a limited regime in which the ratio can significantly distinguish densities.

\FigureThreePDF{figures/tau_ratio_vs_density_thinlimit_sigmavary_Xm10}
               {figures/tau_ratio_vs_density_thinlimit_sigmavary_Xm9}
               {figures/tau_ratio_vs_density_thinlimit_sigmavary_Xm8}
{Optical depth ratios of \ortho \oneone / \twotwo as a function of \dens for three
different abundances.  The different $\sigma$ values represent different widths
of lognormal distributions.
(a) $X=10^{-10}$ (b) $X=10^{-9}$ (c) $X=10^{-8}$ }
{fig:ratiovsdens}


\section{Collisional Coefficients}
The \citet{Troscompt2009a} collisional coefficients only cover the
\formaldehyde energy levels with upper energy $E_U<50$ K. This means they
should not be used above, say, 30 K.  Figure \ref{fig:trosfaur} shows
the optical depths computed using the two different collisional coefficients;
significant discrepancies appear at T$=50K$, $n=10^6 \percc$.

\FigureThreePDF{figures/Faure_Troscompt_compare_T=20}
               {figures/Faure_Troscompt_compare_T=50}
               {figures/Faure_Troscompt_compare_T=80}
{Optical depth as a function of density for 3 different temperatures and 3 different
abundances using the \citet{Troscompt2009a} and \citet{Faure2013a} collision rates.}
{fig:trosfaur}

\end{document}
